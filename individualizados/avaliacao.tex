% Options for packages loaded elsewhere
\PassOptionsToPackage{unicode}{hyperref}
\PassOptionsToPackage{hyphens}{url}
\PassOptionsToPackage{dvipsnames,svgnames,x11names}{xcolor}
%
\documentclass[
  a4paper,
]{article}

\usepackage{amsmath,amssymb}
\usepackage{iftex}
\ifPDFTeX
  \usepackage[T1]{fontenc}
  \usepackage[utf8]{inputenc}
  \usepackage{textcomp} % provide euro and other symbols
\else % if luatex or xetex
  \usepackage{unicode-math}
  \defaultfontfeatures{Scale=MatchLowercase}
  \defaultfontfeatures[\rmfamily]{Ligatures=TeX,Scale=1}
\fi
\usepackage{lmodern}
\ifPDFTeX\else  
    % xetex/luatex font selection
\fi
% Use upquote if available, for straight quotes in verbatim environments
\IfFileExists{upquote.sty}{\usepackage{upquote}}{}
\IfFileExists{microtype.sty}{% use microtype if available
  \usepackage[]{microtype}
  \UseMicrotypeSet[protrusion]{basicmath} % disable protrusion for tt fonts
}{}
\makeatletter
\@ifundefined{KOMAClassName}{% if non-KOMA class
  \IfFileExists{parskip.sty}{%
    \usepackage{parskip}
  }{% else
    \setlength{\parindent}{0pt}
    \setlength{\parskip}{6pt plus 2pt minus 1pt}}
}{% if KOMA class
  \KOMAoptions{parskip=half}}
\makeatother
\usepackage{xcolor}
\usepackage[top=30mm,left=30mm,right=30mm,bottom=30mm]{geometry}
\setlength{\emergencystretch}{3em} % prevent overfull lines
\setcounter{secnumdepth}{5}
% Make \paragraph and \subparagraph free-standing
\ifx\paragraph\undefined\else
  \let\oldparagraph\paragraph
  \renewcommand{\paragraph}[1]{\oldparagraph{#1}\mbox{}}
\fi
\ifx\subparagraph\undefined\else
  \let\oldsubparagraph\subparagraph
  \renewcommand{\subparagraph}[1]{\oldsubparagraph{#1}\mbox{}}
\fi


\providecommand{\tightlist}{%
  \setlength{\itemsep}{0pt}\setlength{\parskip}{0pt}}\usepackage{longtable,booktabs,array}
\usepackage{calc} % for calculating minipage widths
% Correct order of tables after \paragraph or \subparagraph
\usepackage{etoolbox}
\makeatletter
\patchcmd\longtable{\par}{\if@noskipsec\mbox{}\fi\par}{}{}
\makeatother
% Allow footnotes in longtable head/foot
\IfFileExists{footnotehyper.sty}{\usepackage{footnotehyper}}{\usepackage{footnote}}
\makesavenoteenv{longtable}
\usepackage{graphicx}
\makeatletter
\def\maxwidth{\ifdim\Gin@nat@width>\linewidth\linewidth\else\Gin@nat@width\fi}
\def\maxheight{\ifdim\Gin@nat@height>\textheight\textheight\else\Gin@nat@height\fi}
\makeatother
% Scale images if necessary, so that they will not overflow the page
% margins by default, and it is still possible to overwrite the defaults
% using explicit options in \includegraphics[width, height, ...]{}
\setkeys{Gin}{width=\maxwidth,height=\maxheight,keepaspectratio}
% Set default figure placement to htbp
\makeatletter
\def\fps@figure{htbp}
\makeatother
% definitions for citeproc citations
\NewDocumentCommand\citeproctext{}{}
\NewDocumentCommand\citeproc{mm}{%
  \begingroup\def\citeproctext{#2}\cite{#1}\endgroup}
\makeatletter
 % allow citations to break across lines
 \let\@cite@ofmt\@firstofone
 % avoid brackets around text for \cite:
 \def\@biblabel#1{}
 \def\@cite#1#2{{#1\if@tempswa , #2\fi}}
\makeatother
\newlength{\cslhangindent}
\setlength{\cslhangindent}{1.5em}
\newlength{\csllabelwidth}
\setlength{\csllabelwidth}{3em}
\newenvironment{CSLReferences}[2] % #1 hanging-indent, #2 entry-spacing
 {\begin{list}{}{%
  \setlength{\itemindent}{0pt}
  \setlength{\leftmargin}{0pt}
  \setlength{\parsep}{0pt}
  % turn on hanging indent if param 1 is 1
  \ifodd #1
   \setlength{\leftmargin}{\cslhangindent}
   \setlength{\itemindent}{-1\cslhangindent}
  \fi
  % set entry spacing
  \setlength{\itemsep}{#2\baselineskip}}}
 {\end{list}}
\usepackage{calc}
\newcommand{\CSLBlock}[1]{\hfill\break\parbox[t]{\linewidth}{\strut\ignorespaces#1\strut}}
\newcommand{\CSLLeftMargin}[1]{\parbox[t]{\csllabelwidth}{\strut#1\strut}}
\newcommand{\CSLRightInline}[1]{\parbox[t]{\linewidth - \csllabelwidth}{\strut#1\strut}}
\newcommand{\CSLIndent}[1]{\hspace{\cslhangindent}#1}

\let\paragraph\oldparagraph
\let\subparagraph\oldsubparagraph

\usepackage{xcolor}
\usepackage{titlesec}
\usepackage{graphicx}
\usepackage{fancyhdr}

%% Cores nos títulos

\definecolor{headingcolor}{HTML}{00496D} % Blue color

\titleformat{\section}
  {\color{headingcolor}\normalfont\Large\bfseries}
  {\color{headingcolor}\thesection}{1em}{}

\titleformat{\subsection}
  {\color{headingcolor}\normalfont\large\bfseries}
  {\color{headingcolor}\thesubsection}{1em}{}

\titleformat{\subsubsection}
  {\color{headingcolor}\normalfont\normalsize\bfseries}
  {\color{headingcolor}\thesubsubsection}{1em}{}

%% Edita cabeçalho e rodapé

\fancypagestyle{myfooter}{%
  \fancyhf{} % Clear header and footer
  \fancyfoot[L]{onsv.org.br} % Left footer text
  \fancyfoot[R]{\thepage} % Right footer: page number
  \fancyhead[L]{\includegraphics[width=3.5cm]{_extensions/onsv/onsvpub/onsv_logo_header.png}}
  \fancyhead[R]{\includegraphics[width=1.75cm]{_extensions/onsv/onsvpub/obs2030.png}}
  \setlength{\headsep}{0.5in}
  \renewcommand{\headrulewidth}{0.4pt} % Add header rule
  \renewcommand{\footrulewidth}{0.4pt} % Add a footer rule
}

\pagestyle{myfooter}
\usepackage{booktabs}
\usepackage{caption}
\usepackage{longtable}
\usepackage{colortbl}
\usepackage{array}
\usepackage[noblocks]{authblk}
\renewcommand*{\Authsep}{, }
\renewcommand*{\Authand}{, }
\renewcommand*{\Authands}{, }
\renewcommand\Affilfont{\small}
\makeatletter
\@ifpackageloaded{caption}{}{\usepackage{caption}}
\AtBeginDocument{%
\ifdefined\contentsname
  \renewcommand*\contentsname{Índice}
\else
  \newcommand\contentsname{Índice}
\fi
\ifdefined\listfigurename
  \renewcommand*\listfigurename{Lista de Figuras}
\else
  \newcommand\listfigurename{Lista de Figuras}
\fi
\ifdefined\listtablename
  \renewcommand*\listtablename{Lista de Tabelas}
\else
  \newcommand\listtablename{Lista de Tabelas}
\fi
\ifdefined\figurename
  \renewcommand*\figurename{Figura}
\else
  \newcommand\figurename{Figura}
\fi
\ifdefined\tablename
  \renewcommand*\tablename{Tabela}
\else
  \newcommand\tablename{Tabela}
\fi
}
\@ifpackageloaded{float}{}{\usepackage{float}}
\floatstyle{ruled}
\@ifundefined{c@chapter}{\newfloat{codelisting}{h}{lop}}{\newfloat{codelisting}{h}{lop}[chapter]}
\floatname{codelisting}{Listagem}
\newcommand*\listoflistings{\listof{codelisting}{Lista de Listagens}}
\makeatother
\makeatletter
\makeatother
\makeatletter
\@ifpackageloaded{caption}{}{\usepackage{caption}}
\@ifpackageloaded{subcaption}{}{\usepackage{subcaption}}
\makeatother
\ifLuaTeX
\usepackage[bidi=basic]{babel}
\else
\usepackage[bidi=default]{babel}
\fi
\babelprovide[main,import]{brazilian}
% get rid of language-specific shorthands (see #6817):
\let\LanguageShortHands\languageshorthands
\def\languageshorthands#1{}
\ifLuaTeX
  \usepackage{selnolig}  % disable illegal ligatures
\fi
\usepackage{bookmark}

\IfFileExists{xurl.sty}{\usepackage{xurl}}{} % add URL line breaks if available
\urlstyle{same} % disable monospaced font for URLs
\hypersetup{
  pdftitle={Avaliação do Nível de Informações Disponibilizadas no Portal do Detran - TO},
  pdfauthor={João Pedro Melani Saraiva; Pedro Augusto Borges dos Santos; Dr.~Jorge Tiago Bastos},
  pdflang={pt-BR},
  colorlinks=true,
  linkcolor={blue},
  filecolor={Maroon},
  citecolor={Blue},
  urlcolor={Blue},
  pdfcreator={LaTeX via pandoc}}

\title{Avaliação do Nível de Informações Disponibilizadas no Portal do
Detran - TO}


\author[1]{João Pedro Melani Saraiva}
\author[1]{Pedro Augusto Borges dos Santos}
\author[2]{Dr.~Jorge Tiago Bastos}

\affil[1]{Observatório Nacional de Segurança Viária}
\affil[2]{Universidade Federal do Paraná}


\date{2024-09-30}
\begin{document}
\maketitle

\section{Sobre o relatório}\label{sobre-o-relatuxf3rio}

Este trabalho é uma iniciativa do Observatório Nacional de Segurança
Viária, elaborado em parceria com a Universidade Federal do Paraná. Foi
elaborado um documento individual para cada Departamento Estadual de
Trânsito, a fim de expor os resultados para cada instituição.

\section{Introdução}\label{introduuxe7uxe3o}

\subsection{Objetivo}\label{objetivo}

O presente documento tem por objetivo apresentar uma avaliação do nível
de informações disponibilizadas no portal (sítio eletrônico) do Detran
(Departamento Estadual de Trânsito) do Tocantins.

Os Detrans, por sua atuação, trabalham com um grande número de
informações relevantes para a gestão do trânsito no país, tendo muitas
delas, além do teor formal e burocrático, uma grande utilidade para
avaliações no âmbito da segurança viária, tais como: infrações de
trânsito, sinistros de trânsito, e habilitação, entre outros.

Nesse sentido, este documento busca apresentar uma análise da situação
deste Detran, permitindo obter um panorama de como esta instituição
disponibiliza informações de interesse social. Além da introdução, o
documento compõe-se das seguintes seções: Seção~\ref{sec-metodos}, com
as consultas preliminares, pontuação e critérios estabelecidos; e
Seção~\ref{sec-resultados}, com a pontuação geral em cada categoria
analisada, com a evolução das notas resultantes ao longo dos anos.

\subsection{Atribuições do Detran}\label{atribuiuxe7uxf5es-do-detran}

Conforme o CTB (Código de Trânsito Brasileiro), são competências do
Detran o planejamento, coordenação, execução e controle relacionados às
seguintes áreas: habilitação de condutores, veículos, fiscalização de
trânsito, estatísticas e educação para o trânsito.

Na habilitação de condutores, cabe ao Detran realizar e controlar o
processo de formação, reciclagem e suspensão dos condutores, assim como
a expedição e cassação da Licença de Aprendizagem, Permissão para
Dirigir e CNH.

Em relação aos veículos, compete ao Detran o seu registro, emplacamento,
licenciamento e a expedição do seu Certificado de Registro e
Licenciamento Anual.

Na área de fiscalização, o Detran deve fazer cumprir a legislação e as
normas de trânsito, sendo responsável pela autuação e aplicação de
medidas administrativas cabíveis pelas infrações previstas no Código de
Trânsito Brasileiro.

Em relação à educação para o trânsito, é responsabilidade do Detran a
promoção e participação de projetos de educação e segurança no trânsito.
Por fim, coleta de dados estatísticos e a elaboração de estudos sobre
acidentes de trânsito e suas causas também devem ser cumpridos pelo
Detran.

\subsection{Versão anterior}\label{versuxe3o-anterior}

Este trabalho faz parte da segunda versão do relatório anteriormente
publicado em 2020 pelo Observatório Nacional de Segurança Viária, que
virou uma comunicação técnica publicada no segundo Simpósio de
Transportes do Paraná (STPR), também em (2020). A elaboração de uma nova
versão irá possibilitar a evolução dos portais dos Detrans de cada
estado.

\section{Métodos}\label{sec-metodos}

\subsection{Consultas Preliminares}\label{consultas-preliminares}

O estudo foi realizado por meio do acesso ao \emph{site} do Detran-TO a
fim de averiguar a existência de dados estatísticos sobre a frota
veicular, condutores habilitados, infrações e sinistros de trânsito.
Também foram avaliadas a existência de informações relacionadas à
educação para o trânsito (campanhas, material educativo, regras de
trânsito, etc.), dados quantitativos do atendimento ao público pelo
departamento, disponibilidade de canais de atendimento e dados
informativos sobre os centros de formação de condutores do estado em
questão. A Tabela~\ref{tbl-consultas} demonstra os resultados de uma
consulta preliminar destes portais

\begin{longtable}{lc}

\caption{\label{tbl-consultas}Consultas preliminares}

\tabularnewline

\toprule
Categoria & Situação \\ 
\midrule\addlinespace[2.5pt]
CFCs & \cellcolor[HTML]{1FA149}{\textcolor[HTML]{FFFFFF}{Presente}} \\ 
Nota Final & \cellcolor[HTML]{1FA149}{\textcolor[HTML]{FFFFFF}{Presente}} \\ 
Frota & \cellcolor[HTML]{D7191C}{\textcolor[HTML]{FFFFFF}{Ausente}} \\ 
Condutores & \cellcolor[HTML]{D7191C}{\textcolor[HTML]{FFFFFF}{Ausente}} \\ 
Infrações & \cellcolor[HTML]{D7191C}{\textcolor[HTML]{FFFFFF}{Ausente}} \\ 
Sinistros & \cellcolor[HTML]{D7191C}{\textcolor[HTML]{FFFFFF}{Ausente}} \\ 
Atendimento & \cellcolor[HTML]{1FA149}{\textcolor[HTML]{FFFFFF}{Presente}} \\ 
Educação & \cellcolor[HTML]{1FA149}{\textcolor[HTML]{FFFFFF}{Presente}} \\ 
\bottomrule

\end{longtable}

\subsection{Critérios de
Avaliação}\label{crituxe9rios-de-avaliauxe7uxe3o}

A partir das categorias anteriormente discutidas, critérios de avaliação
foram estabelecidos para quantificar a disponibilidade de dados e
informações de cada um dos Detrans. Assim sendo, são concedidos níveis
de desempenho que equivalem a pontuações para cada um dos critérios
aferidos:

\begin{enumerate}
\def\labelenumi{\arabic{enumi}.}
\item
  \textbf{Melhor Prática} - Nota 10;
\item
  \textbf{Prática Intermediária} - Nota 6,7;
\item
  \textbf{Prática Inicial} - Nota 3,3 e;
\item
  \textbf{Prática Ausente} - Nota 0.
\end{enumerate}

A avaliação dos dados estatísticos da \textbf{frota veicular},
\textbf{condutores habilitados}, \textbf{infrações} e \textbf{sinistros}
de trânsito se baseou nos critérios de periodicidade (quantidade de anos
disponíveis), interatividade (tipo de arquivos ou informações
disponíveis) e nível de desagregação (quantidade de variáveis
disponíveis para análise).

A \textbf{periodicidade} foi avaliada em:

\begin{itemize}
\item
  Melhor: Quando há um máximo de 2 anos de atraso em relação ao ano
  atual para o dado mais recente e mais de 5 anos de dados disponíves;
\item
  Intermediária: Máximo de 2 anos de atraso ou menos de 5 anos
  disponíveis e;
\item
  Inicial: 3+ anos de atraso e menos de 5 anos disponíveis.
\end{itemize}

Neste contexto, para um estado ser situado como \emph{Melhor}, por
exemplo, seria necessário ter mais de 5 anos disponíveis e seu dado mais
recente ser de, no mmínimo 2022.

A \textbf{interatividade} foi avaliada em:

\begin{itemize}
\item
  Melhor: Possui dados disponibilizados em relatórios interativos ou
  \emph{dashboards} para consultas dinâmicas;
\item
  Intermediária: Dados disponibilizados em arquivos de planinhas (como
  \texttt{.xls}, \texttt{.xlsx} e \texttt{.ods}) de editores como
  Microsoft Excel, Google Sheets e LibreOffice Calc, e;
\item
  Inicial: Dados disponibilizados apenas como texto na página ou
  arquivos de texto (\texttt{.txt}, \texttt{.csv}, etc.).
\end{itemize}

O \textbf{nível de desagregação} foi avaliado em:

\begin{itemize}
\item
  Melhor: Quantidade grande de variáveis;
\item
  Intermediário: Quantidade média de variáveis e;
\item
  Inicial: Quantidade baixa de variáveis.
\end{itemize}

A relação destes critérios está ilustrada na Tabela~\ref{tbl-criterios}:

\break

\begin{longtable}{l|cccc}

\caption{\label{tbl-criterios}Critérios de avaliação para frota,
condutores habilitados, infrações e sinistros}

\tabularnewline

\toprule
\multicolumn{1}{l}{} & Periodicidade & Interatividade & Nível de desagregação & Notas \\ 
\midrule\addlinespace[2.5pt]
Frota & 3+ anos de atraso e < 5 anos & Texto & 1 a 2 desagregações & 3.3 \\ 
 & 2 anos de atraso ou < 5 anos & Planilha & 3 a 4 desagregações & 6.7 \\ 
 & 2 anos de atraso e > 5 anos & Interativo & 5+ desagregações & 10.0 \\ 
\midrule\addlinespace[2.5pt]
Condutores & 3+ anos de atraso e < 5 anos & Texto & 1 a 2 desagregações & 3.3 \\ 
 & 2 anos de atraso ou < 5 anos & Planilha & 3 a 4 desagregações & 6.7 \\ 
 & 2 anos de atraso e > 5 anos & Interativo & 5+ desagregações & 10.0 \\ 
\midrule\addlinespace[2.5pt]
Infrações & 3+ anos de atraso e < 5 anos & Texto & 1 desagregação & 3.3 \\ 
 & 2 anos de atraso ou < 5 anos & Planilha & 2 desagregações & 6.7 \\ 
 & 2 anos de atraso e > 5 anos & Interativo & 3+ desagregações & 10.0 \\ 
\midrule\addlinespace[2.5pt]
Sinistros & 3+ anos de atraso e < 5 anos & Texto & 1 a 5 desagregações & 3.3 \\ 
 & 2 anos de atraso ou < 5 anos & Planilha & 6 a 10 desagregações & 6.7 \\ 
 & 2 anos de atraso e > 5 anos & Interativo & 11+ desagregações & 10.0 \\ 
\bottomrule

\end{longtable}

As informações sobre o \textbf{atendimento ao público}, \textbf{educação
para o trânsito} e \textbf{CFCs} foram aferidas com outros critérios
mais específicos por não se tratarem de dados estritamente estatísticos,
exigindo uma mudança na abordagem de avaliação. As Tabelas
\ref{tbl-cri-atend}, \ref{tbl-cri-educ} e \ref{tbl-cri-cfcs} demonstram
os critérios para cada um destes casos:

\begin{longtable}{>{\centering\arraybackslash}p{\dimexpr 0.3\linewidth-2\tabcolsep-1.5\arrayrulewidth}>{\centering\arraybackslash}p{\dimexpr 0.3\linewidth-2\tabcolsep-1.5\arrayrulewidth}>{\centering\arraybackslash}p{\dimexpr 0.3\linewidth-2\tabcolsep-1.5\arrayrulewidth}}

\caption{\label{tbl-cri-atend}Critérios de avaliação para atendimento ao
público}

\tabularnewline

\toprule
Periodicidade & Canais de Atendimento & Notas \\ 
\midrule\addlinespace[2.5pt]
3+ anos de atraso e < 5 anos & Atendimento por telefone & 3.3 \\ 
2 anos de atraso ou < 5 anos & Atendimento por telefone e e-mail & 6.7 \\ 
2 anos de atraso e > 5 anos & Atendimento por telefone, e-mail e mensagem & 10.0 \\ 
\bottomrule

\end{longtable}

\begin{longtable}{>{\centering\arraybackslash}p{\dimexpr 0.3\linewidth-2\tabcolsep-1.5\arrayrulewidth}>{\centering\arraybackslash}p{\dimexpr 0.3\linewidth-2\tabcolsep-1.5\arrayrulewidth}}

\caption{\label{tbl-cri-educ}Critérios de avaliação para conteúdo
educativo}

\tabularnewline

\toprule
Conteúdos didáticos & Divulgação de atividade \\ 
\midrule\addlinespace[2.5pt]
Dicas, sugestões de leitura, cartilhas, orientações gerais e cadernos pedagógicos. & Cursos, palestras, ações, projetos e campanhas. \\ 
\bottomrule

\end{longtable}

\begin{longtable}{>{\centering\arraybackslash}p{\dimexpr 0.5\linewidth-2\tabcolsep-1.5\arrayrulewidth}>{\centering\arraybackslash}p{\dimexpr 0.5\linewidth-2\tabcolsep-1.5\arrayrulewidth}}

\caption{\label{tbl-cri-cfcs}Critérios de avaliação para informações
sobre CFCs}

\tabularnewline

\toprule
Lista sobre CFCs credenciados & Notas \\ 
\midrule\addlinespace[2.5pt]
Quantidade de CFCs credenciados no estado & 3.3 \\ 
Lista completa dos CFCs no estado & 6.7 \\ 
Informações adicionais sobre o índice de aprovação dos alunos, quais CFCs possuem simuladores de direção, número de habilitados e informações sobre o credenciamento. & 10.0 \\ 
\bottomrule

\end{longtable}

\section{Resultados}\label{sec-resultados}

A fim de analisar os índices atribuídos para cada tipo de informação
consultada, compara-se os dados de 2020 e 2024. A nota resultante para
cada categoria considerada é calculada por meio da média das pontuações
de seus critérios, como aponta Tabela~\ref{tbl-media}.

\begin{longtable}{l|ccccccc}

\caption{\label{tbl-media}Notas médias para cada categoria}

\tabularnewline

\toprule
\multicolumn{1}{l}{} & CFCs & Frota & Condutores & Infrações & Sinistros & Atendimento & Educação \\ 
\midrule\addlinespace[2.5pt]
2024 & \cellcolor[HTML]{F5951E}{\textcolor[HTML]{000000}{$6,7$}} & \cellcolor[HTML]{D7191C}{\textcolor[HTML]{FFFFFF}{$0,0$}} & \cellcolor[HTML]{D7191C}{\textcolor[HTML]{FFFFFF}{$0,0$}} & \cellcolor[HTML]{D7191C}{\textcolor[HTML]{FFFFFF}{$0,0$}} & \cellcolor[HTML]{D7191C}{\textcolor[HTML]{FFFFFF}{$0,0$}} & \cellcolor[HTML]{F47B20}{\textcolor[HTML]{FFFFFF}{$5,0$}} & \cellcolor[HTML]{F47B20}{\textcolor[HTML]{FFFFFF}{$5,0$}} \\ 
\midrule\addlinespace[2.5pt]
2020 & \cellcolor[HTML]{1FA149}{\textcolor[HTML]{FFFFFF}{$10,0$}} & \cellcolor[HTML]{F5841F}{\textcolor[HTML]{FFFFFF}{$5,6$}} & \cellcolor[HTML]{D7191C}{\textcolor[HTML]{FFFFFF}{$0,0$}} & \cellcolor[HTML]{D7191C}{\textcolor[HTML]{FFFFFF}{$0,0$}} & \cellcolor[HTML]{D7191C}{\textcolor[HTML]{FFFFFF}{$0,0$}} & \cellcolor[HTML]{F47B20}{\textcolor[HTML]{FFFFFF}{$5,0$}} & \cellcolor[HTML]{D7191C}{\textcolor[HTML]{FFFFFF}{$0,0$}} \\ 
\bottomrule

\end{longtable}

A pontuação final de 2024 para o estado em destaque foi computada por
meio da média destas notas, atingindo um valor de \textbf{2,39}, o que
representa um regresso em relação a 2020.

O resultado final das consultas, efetuadas em 2020 e 2024, sobre todas
as categorias e critérios considerados para a avaliação da
disponibilidade de dados no portal do Detran-TO está disponível na
Tabela~\ref{tbl-final}.

\begin{longtable}{l|ccc}

\caption{\label{tbl-final}Resultados finais das consultas}

\tabularnewline

\toprule
\multicolumn{1}{l}{} &  & \multicolumn{2}{c}{Ano} \\ 
\cmidrule(lr){3-4}
\multicolumn{1}{l}{} & Critério & 2020 & 2024 \\ 
\midrule\addlinespace[2.5pt]
Frota & Periodicidade & \cellcolor[HTML]{F5951E}{\textcolor[HTML]{000000}{$6,7$}} & \cellcolor[HTML]{D7191C}{\textcolor[HTML]{FFFFFF}{$0,0$}} \\ 
 & Interatividade & \cellcolor[HTML]{F05E22}{\textcolor[HTML]{FFFFFF}{$3,3$}} & \cellcolor[HTML]{D7191C}{\textcolor[HTML]{FFFFFF}{$0,0$}} \\ 
 & Nível de desagregação & \cellcolor[HTML]{F5951E}{\textcolor[HTML]{000000}{$6,7$}} & \cellcolor[HTML]{D7191C}{\textcolor[HTML]{FFFFFF}{$0,0$}} \\ 
\midrule\addlinespace[2.5pt]
Condutores & Periodicidade & \cellcolor[HTML]{D7191C}{\textcolor[HTML]{FFFFFF}{$0,0$}} & \cellcolor[HTML]{D7191C}{\textcolor[HTML]{FFFFFF}{$0,0$}} \\ 
 & Interatividade & \cellcolor[HTML]{D7191C}{\textcolor[HTML]{FFFFFF}{$0,0$}} & \cellcolor[HTML]{D7191C}{\textcolor[HTML]{FFFFFF}{$0,0$}} \\ 
 & Nível de desagregação & \cellcolor[HTML]{D7191C}{\textcolor[HTML]{FFFFFF}{$0,0$}} & \cellcolor[HTML]{D7191C}{\textcolor[HTML]{FFFFFF}{$0,0$}} \\ 
\midrule\addlinespace[2.5pt]
Infrações & Periodicidade & \cellcolor[HTML]{D7191C}{\textcolor[HTML]{FFFFFF}{$0,0$}} & \cellcolor[HTML]{D7191C}{\textcolor[HTML]{FFFFFF}{$0,0$}} \\ 
 & Interatividade & \cellcolor[HTML]{D7191C}{\textcolor[HTML]{FFFFFF}{$0,0$}} & \cellcolor[HTML]{D7191C}{\textcolor[HTML]{FFFFFF}{$0,0$}} \\ 
 & Nível de desagregação & \cellcolor[HTML]{D7191C}{\textcolor[HTML]{FFFFFF}{$0,0$}} & \cellcolor[HTML]{D7191C}{\textcolor[HTML]{FFFFFF}{$0,0$}} \\ 
\midrule\addlinespace[2.5pt]
Sinistros & Periodicidade & \cellcolor[HTML]{D7191C}{\textcolor[HTML]{FFFFFF}{$0,0$}} & \cellcolor[HTML]{D7191C}{\textcolor[HTML]{FFFFFF}{$0,0$}} \\ 
 & Interatividade & \cellcolor[HTML]{D7191C}{\textcolor[HTML]{FFFFFF}{$0,0$}} & \cellcolor[HTML]{D7191C}{\textcolor[HTML]{FFFFFF}{$0,0$}} \\ 
 & Nível de desagregação & \cellcolor[HTML]{D7191C}{\textcolor[HTML]{FFFFFF}{$0,0$}} & \cellcolor[HTML]{D7191C}{\textcolor[HTML]{FFFFFF}{$0,0$}} \\ 
\midrule\addlinespace[2.5pt]
Atendimento & Estatística & \cellcolor[HTML]{D7191C}{\textcolor[HTML]{FFFFFF}{$0,0$}} & \cellcolor[HTML]{D7191C}{\textcolor[HTML]{FFFFFF}{$0,0$}} \\ 
 & Canais & \cellcolor[HTML]{1FA149}{\textcolor[HTML]{FFFFFF}{$10,0$}} & \cellcolor[HTML]{1FA149}{\textcolor[HTML]{FFFFFF}{$10,0$}} \\ 
\midrule\addlinespace[2.5pt]
Educação & Conteúdos & \cellcolor[HTML]{D7191C}{\textcolor[HTML]{FFFFFF}{$0,0$}} & \cellcolor[HTML]{F05E22}{\textcolor[HTML]{FFFFFF}{$3,3$}} \\ 
 & Divulgação & \cellcolor[HTML]{D7191C}{\textcolor[HTML]{FFFFFF}{$0,0$}} & \cellcolor[HTML]{F5951E}{\textcolor[HTML]{000000}{$6,7$}} \\ 
\midrule\addlinespace[2.5pt]
CFCs & CFCs & \cellcolor[HTML]{1FA149}{\textcolor[HTML]{FFFFFF}{$10,0$}} & \cellcolor[HTML]{F5951E}{\textcolor[HTML]{000000}{$6,7$}} \\ 
\midrule\addlinespace[2.5pt]
Nota Final & Nota Final & \cellcolor[HTML]{ED5921}{\textcolor[HTML]{FFFFFF}{$2,9$}} & \cellcolor[HTML]{E94F20}{\textcolor[HTML]{FFFFFF}{$2,4$}} \\ 
\bottomrule

\end{longtable}

\section{Conclusão}\label{conclusuxe3o}

O presente trabalho avaliou de forma quantitativa o desempenho do portal
do departamento consultado em disponibilizar ao público informações
pertinentes à segurança viária do respectivo estado. Desta maneira, a
categoria de melhor desempenho para o Detran do Tocantins foi a de CFCs,
com nota igual a 6,7. Em contraparte, as categorias de pior desempenho
foram as de Frota, Condutores, Infrações e Sinistros, com nota 0.

Os resultados discutidos nesta análise visam não apenas mensurar o
desempenho do fornecimento de dados, mas também auxiliar na instrução de
possíveis melhorias para o sítio eletrônico estudado, visando agregar à
qualidade e disponibilidade das informações providenciadas pelo
Detran-TO em conformidade às suas atribuições.

\section*{Referências}\label{referuxeancias}
\addcontentsline{toc}{section}{Referências}

\phantomsection\label{refs}
\begin{CSLReferences}{0}{1}
\bibitem[\citeproctext]{ref-santos2020}
SANTOS, P. A. B. DOS; BASTOS, J. T.; GARONCE, F. V. \textbf{AVALIAÇÃO DO
CONTEÚDO DOS PORTAIS DOS DEPARTAMENTOS ESTADUAIS DE TRÂNSITO - DETRANs}.
28 nov. 2020. Disponível em:
\textless{}\url{https://stpr.ufpr.br/wp-content/uploads/2021/02/Anais-STPR-2020-1.pdf}\textgreater{}

\end{CSLReferences}



\end{document}
